\section{Önsöz}

Şiir neden yazılır? Şu bulunduğumuz noktaya dek ne acayip akımlar geçti. Şiir sözlere ağır geldi kağıda döküldü, parçalara bölündü, irili ufaklı ikilemelere büründü, tekrar birleşti, dağıldı etti. Ne oldu sonunda? Başladığı noktaya geri döndü; sözlerin yapısına ayak uydurdu. Yazarlara da ölmek kaldı mezarlarında...
İnsan ne yapıp ettiyse de yazma ihtiyacından bir türlü kurtulamadı. Bunun bir yetenek olduğunu sanmıyorum, her insan önüne kağıdı alıp bir şeyler karalayabilir. Yazmak daha doğrusu bir ihtiyaç; insanın varoluşsal derdinden doğan bir dert. Hepimizin hayatında, günlük yaşamımızda dahi, üstümüzden atamadığımız bir mahzunluk hali var. Orhan Veli de bunu zamanında Garip Akımı'nda belirtmiş. İnsanlığın ortak olarak paylaştığı bu mahzunluk hali insanı yazmaya yönelten yegane şeydir. İnsan yazmayı bir varoluş meselesi olarak düşünüp sanki yok olurcasına yazı yazmalıdır. Şiiri kendisi için yazmamalı, hissettikleri için yazmalıdır.

\noindent\newline
Evet belki daha çok bilgisiz ve deneyimsizim; ama size bahsedeyim neden yazdığımı. Hani bazı durumlarda saçma sapan bir işle uğraşırsınız; aynı şeyi defalarca tekrarladığınız türden. Böyle durumlarda yaptığınız el hareketleri bir mertebeden sonra kendi başlarına devam etmeye başlar. Sanki siz hiç çaba harcamazsınız devam ettirmek için. İşte böyle durumlarda içinde bulunduğunuz durumun dışına çıkarsınız, farklı bir düşünce aleminin içine dalarsınız. Bilirsiniz ki o an gördüğünüz işi yapmaktasınız; fakat aklınızdan geçen düşünceler çoğaldıkça gözlerinizin önüne acayip görüntüler gelir. Aynı olayı defalarca yaşamışımdır, hem de farklı suretlerde. Bazen sandalyede yorgun bir halde oturup elime defteri alırım. Yorgunluğun da etkisi altında beyin sanki aktif bir düşünme işleminden ziyade pasif bir duruma gelir. Böyle anlarda adeta düşünce aleminden beynime düşen düşünce parçaları olurcasına kalem hareket etmeye başlar ve cümleler yazılır. Her şeyin farkındasınızdır; ama yine de bir şekilde düzenin dışına çıktığınızı hissedersiniz. Benliğiniz ortamında içinde kaybolurken siz "dışarıki" bir odada olduğunuzu hissedersiniz.

\noindent\newline
Hele ki bu pasif düşünce halinde aklınıza gelen yahut kağıtta yazılı bir dize okursunuz. En sevdiğiniz şairdendir veya günün birinde birinden duymuşsunuzdur. İşte tam o sırada yazılı söz size bir anda çarpar. Sırtınızdan, omuriliğiniz boyunca beyninize doğru bir uyarılma hissedersiniz, tüyleriniz diken diken olur. İşte tam da bu duygu nedeniyle şiir yazıyor ve okuyorum. Bana sorarsanız her hareketimizin nedeni hissettiğimiz duygulardır. Duygulara yenik düşüp hareket etmemiz bir o kadar ilkel ve vahşi hayvanları andırırcasına olsa da bu gerçeği göz önünden çıkarmamamız gerekiyor. Ne kadar modern insanlar olarak analitik düşünmeye yöneldiysek de insani ilişkiler içersinde bu duygusal birikimimizden sıyrılmamamız gerekiyor. Zaten hiçbir bilginin doğruluğunu teyit edemeseniz bile hissettiğiniz duygu o an en gerçek olanıdır.

\noindent\newline
Hayatım boyunca birçok yanlış yaptım, ihanete uğradım, sayısız pişmanlıklar yaşadım; ama her zaman ne hissettiğimin farkına varmaya çalıştım. Şiirde de düşüncelerden çok hislere yönelmeyi doğru buldum; çünkü hislerin doğruluğundan kesin olarak emin olabiliyordum. İnsanın üstünden atamadığı o mahzunluk hissi de bu nedenle şiirden hiç eksilmedi. Belki de insan olmanın bir teyidi olarak bu duyguyu kullanmamız gerekiyordu.
Tekrar dönüp yazılanlara baktığımda zaman içinde ne kadar saçmaladığımın farkına varıyorum. Sürekli anıları didiklemekle uğraştım fakat unuttuğum bir şey vardı ki hatıraların ucu bir yere çıkmıyor. İnsan nerede ne halde olsa da geçmişi anmak kendisini o durumdan kurtaramıyor. Şiiri o anlattığım pasif düşünce haline benzetecek olursak kendisine has bir işleyişi var. Bilmelisiniz ki şiirde her şey yaşanabilir. O düşünce parçaları doğru şekilde birleşirse gereken duyguya bürünebilir. Durumun böyle olduğunu kabul edecek olursak şiir yazana değil ihtiyacı olana aittir. En azından Pablo Neruda böyle söylüyordu.

\noindent\newline
Neticede şiirin varlığının temeli sizde yarattığı duygular olmalıdır. Eğer okuduğunuz bir dize sizde ihtiyaç duyduğunuz duyguları yaratamıyorsa amacına ulaşamamış demektir. Duygulara karşın üzerinde uzunca durduğunuz düşünceler bile sonunda bir anlam yaratabilir.

\begin{flushright}
    Kaan Giray Buzluk
\end{flushright}