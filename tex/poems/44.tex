\subsection{Yetişkin}

Kimsesiz bir çocukluk \\
Yatağının başucunda boş evin tek yalnızı \\
Çekmecesindeki oyuncaklarla oynamasına bi çare \\
Kendi kurduğu oyunlardan kovulması \\
Oysa oyuncaklar yaşlarını bilmez \\
Ve de olgunlaşmazlar zamanla \\
Ancak solmakta olur renkleriyle \\
Yüzlerdeki değişmez ifadeler

\noindent\newline
Abim olsun da istemedim hiç \\
Tanımadığım amcalar, yengelerle büyüdüm \\
Seneler boyu tek bir kelime duymadan \\
Hangi birine sorsanız uzaktan yakınımdılar ama \\
Uyuyana kadar başucumda kimse durmadan

\noindent\newline
Ne güzel olurdu oysa \\
Öğretmeselerdi bir maarifetmiş gibi \\
Kimmiş o dillerden düşmez Ayşe \\
Ve top tutmaktaki Ali \\
Büyüyünce ben de öğrendim adam akıllı \\
Yetişkinler gibi kin tutmayı, yalan söylemeyi \\
Bir iplik ardında eskitmek seneleri \\
Alındığı unutulmuş ceketlerde \\
Farkına varılmamış sökükler gibi

\noindent\newline
Artık kalemler kırıldı \\
İdamım kesinleşmiş \\
Çocuğu boğ diyorlar kendi ellerinle \\
Çocuk güler gibi ağlıyor \\
Hatırladıkça dizine bakıp sokaklarda düştüğü yerleri \\
Unutmasını söylüyorum geçmişi \\
Güler gibi ağlıyor çocuk \\
Ve ben eski günleri anıyorum \\
Sükûneti mühürlüyor ama son bir sözle, aşıyor bendimi: \\
	"Herkes rahatça ayrılabilirdi sanki, \\
		insan ölümü bir kez yendi mi."