\subsection{Bir Varmış Bir Yokmuş}

Yürüyüp de sonunu getiremediğimiz \\
o sahilin başlangıcını kimse sormadı \\
Herkes çünkü ortasında başlamıştı \\
Şimdi anlıyorum ne kadar masum olduğunu kalbin \\
Çünkü o görmeden, duymadan, tatmadan bağlanır \\
Aşkın sarhoş edici sıcaklığına \\
Ve bırakamaz bu yüzden \\
Zaten başında kabul saymadığı o devin varlığını. \\

\noindent\newline
Bazıları fabllara inanır \\
Eşekler konuşur insan diliyle eşek gibi \\
Diğerleri de peri masallarına inanır \\
İnkar yok biz de inanmıştık \\
Kötü başlayan bir hikayenin güzel biteceğine \\
Hikayemizin bir peri masalı olacağına \\

\noindent\newline
Geceleri yağmur yağacaktı herkes evdeyken \\
Camlarda hep bir ağızdan damlaların senfoni, \\
Gündüzleriyse güneş doğacaktı, \\
Ve hiç batmadan bir kez daha doğacaktı, \\
Ve bir kez daha kendi gölgesini aldatarak \\
Ne olduysa olması gerekti \\
Çocuklar mucizelere inanmasın diye \\
Bütün sayfaları yırtıp attım. \\

\noindent\newline
Şimdi gözlerin denize meydan okusa kim yazar? \\
Hiç bir sahilin sonuna yürüyüp de ulaştık mı? \\
Ya da bellemiş miydik ayın gökyüzünde yükselişini, \\
Gözümüzden yaşlar akarken? \\

\noindent\newline
Hiçbir önemi yok artık \\
Çiçekler açmıyor diye onları sulamıyorum, \\
Eski günleri arasam da gelecekte bulamıyorum, \\
Aşkın bir dönemi yok artık \\
Hepsi hatıralar ve acı. \\

\noindent\newline
Ama yine de hatırlamaya mahkumum \\
Yüzüm unutsa bile yüzünü \\
Ellerim asla unutmayacak; \\
Dokunduğu her tende yaratmaya çalışacak \\
Mermeri elleriyle yontan bir heykeltıraş gibi, \\
	hüzünü. \\