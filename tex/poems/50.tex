\subsection{21. Yüzyıl Tragedyası}

Ne açlıktır derdi onun \\
Ne de eline alıp dokunamadığı toprak \\
Kısılmış gözlerle bakar gökyüzüne \\
Ve sanki kaymayan yıldızlardan iğrenircesine \\
Gözlerini ovalar, ağlayarak \\

\noindent\newline
Hakkı tabi ama insan olmak, insan olmadan hem de \\
Kaybolması gibi yedi milyar kişinin \\
Tek bir kişi içinde \\
Burda ilahi adalet ne arar, haydi git duyur \\
Gelip toplansınlar eğri büğrü bir top üstüne \\
Söyle onlara ki en kötüler, en günahkarlar bile \\
Gece yataklarında cehennemlerde yanışı düşünmeden \\
Vicdanları hür uyur \\

\noindent\newline
Bir utanç kaleminden çıkmış gibi bu dünya \\
Kahramanların kanla yazılı hikayeleri bile ortadan yırtılır \\
Hayattan geriye gerçi kahramanlar da kalmaz \\
Herkes daha bir iyi, daha bir asil görünür burda ama \\
Her yansımaya düşecek ayna bulunmaz \\

\noindent\newline
Elinde bir kova suyla ne yapacaksın \\
Sakın onu alıp da yerlere dökeyim deme \\
Çiçek sulandıkça hapsolur çünkü bulunduğu yere \\
Sen ise hapsoldukça sulanırsın \\

\noindent\newline
Her şeye sıra gelir, yeter ki bekle \\
Ölmeye bile izin var, hiç doğmamışsan eğer \\
Fakat sen de seviyorsan diğer herkes gibi çiçekleri \\
Hiçbir anlam ifade etmez solması dalındaki gülün \\
Ve konması on yıllar sonra \\
Bir garibanın naaşı üstüne \\

\noindent\newline
Ne açlıktır derdi onun \\
Ne de doğumdan beri hasret kalması toprağa \\
Belki tekrarı bile oynanır gelecek yüzyıllarda \\
En güçlümüz, en yiğidimiz bile \\
Kırık bir toka karşısında ağlayakalır \\
Üç beş cümle, yahut bir isim bile etmez bin yıl süren yası \\
Alın size işte yirmibirinci yüzyıl tragedyası. \\